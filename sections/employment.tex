\section{Employment}
\vspace{0.5em}

\affiliation[Research Associate]{University of Colorado, Boulder,
  CO}{2013--present}
\begin{compactitem}[\itembullet]
  \item
    Performed duties of Research Software Engineer in the Community
    Surface Dynamics Modeling System (CSDMS) Integration Facility at
    the Institute of Arctic and Alpine Research (INSTAAR).
  \item
    Co-developed the \textit{CSDMS Python Modeling Toolkit} (PyMT), an
    open source Python package that provides the tools needed to run
    and couple models that expose the Basic Model Interface (BMI).
  \item
    Co-developed the \textit{CSDMS Web Modeling Tool} (WMT), a RESTful
    web application that allows users to run standalone or coupled
    models on a high-performance computing system. Wrote the web
    client with GWT in Java. Co-wrote the backend server with Python
    and SQLite.
  \item
    Developed \textit{Dakotathon}, a scriptable Python interface to
    the Dakota iterative systems analysis toolkit.
  \item    
    Developed the \textit{Permafrost Benchmark System} (PBS), which
    wraps the command-line ILAMB benchmarking software with a
    customized version of WMT, and adds Python-based tools for
    uploading CMIP5-compatible model outputs and benchmark datasets.
  \item
    Developed Java and Fortran implementations of the \textit{CSDMS
      Basic Model Interface} (BMI).
  \item
    Co-developed the \textit{CSDMS Bakery}, a repository of Anaconda recipes
    for model components, tools, and core cyberinfrastructure.
  \item
    Developed tools for building and installing CSDMS software on
    RPM-based Linux distributions.
  \item Invited talks:
    \begin{compactitem}[\itembullet]
      \item
        2019 USGS Community for Data Integration (CDI) Workshop
      \item
        2018 Coupling of Tectonic and Surface Processes (CTSP)
        Workshop
    \end{compactitem}
  \item Co-Investigator on five funded projects:
    \begin{compactitem}[\itembullet]
      \item
        USGS CDI 2020: Research to Operations: Ensemble Modeling at
        Hillslope to Regional Scales
      \item
        NSF-1924259: Cybertraining: Pilot: Collaborative Research:
        Cybertraining for Earth Surface Processes Modelers
      \item
        USGS CDI 2019: Coupling Hydrologic Models with Data Services
        in an Interoperable Modeling Framework
      \item
        NSF-1503559: Toward a Tiered Permafrost Modeling Cyberinfrastructure
      \item
        NASA-14-CMAC14-0014: A Permafrost Benchmark System to Evaluate
        Permafrost Models
    \end{compactitem}
  \item
    Taught Software Carpentry workshops with customized content for
    Earth surface processes modeling.
  \item
    Taught workshops on programming using tools developed by CSDMS for
    integrating numerical models into the CSDMS modeling framework.
  \item
    Co-convener of the annual \textit{BMI Hackathon} to help CSDMS
    community members add a Basic Model Interface to their model.
  \item
    Served as a judge for the CSDMS Student Modeler Award.
  \item Internal programming resource for the broader CSDMS community.
\end{compactitem}

\affiliation[Product Manager]{L3Harris Geospatial Solutions
  (formerly Research Systems, Inc. (RSI)), Boulder, CO}{2012--2013}
\begin{compactitem}[\itembullet]
  \item Technical product manager, product owner, and technology
    evangelist for the scientific data analysis and visualization
    language IDL. In concert with the lead developer, tasked with
    guiding the overall direction of the language, including
    determining what new capabilities to add and prioritizing what
    bugs to fix.
  \item Performed market research and competitive analyses (versus,
    primarily, Python and MATLAB), gathering user feedback though
    onsite visits (particularly at NASA, NOAA, and U.S. National Labs)
    and attendance at scientific conferences, participating in the
    design of the IDL and ENVI (geospatial image analysis application)
    APIs, writing user stories, developing requirements, managing the
    product backlog and effectively communicating information to the
    Engineering team.
  \item Acted as an internal technical resource for Sales, Marketing,
    Tech Support, Services, and Engineering.
\end{compactitem}

\affiliation[Solutions Engineer]{L3Harris Geospatial
  Solutions, Boulder, CO}{2010--2012}
\begin{compactitem}[\itembullet]
  \item Acted as product owner for IDL. Duties included writing user
    stories and translating user feedback to product development,
    including working with the IDL product manager and lead developer
    to prioritize development for future releases. Responsible for
    internal and external technical IDL product launch communication,
    using presentations, whitepapers, and code to demonstrate new
    features and functionality.
  \item Acted as technology evangelist for IDL. Attended scientific
    conferences to meet customers, obtain product feedback, and
    conduct market research. Engaged in user outreach by developing
    and presenting quarterly live seminars and webinars on topics
    ranging from numerical analysis techniques to file formats to
    visualization techniques in IDL. Created and maintained a blog,
    highlighting solutions to current problems with IDL. Engaged with
    users through Twitter (@TheIDLGuy). Wrote programmatic examples
    for customers. Acted as a resource to the IDL user community
    through communication with long-time customers and through
    involvement with the \url{comp.lang.idl-pvwave} Usenet newsgroup.
  \item Continued to act as an internal technical resource for Sales,
    Marketing, Tech Support, Services, Engineering, and Product
    Management.
\end{compactitem}

\affiliation[Professional Services Engineer]{L3Harris Geospatial
  Solutions, Boulder, CO}{1999--2010}
\begin{compactitem}[\itembullet]
  \item Designed, developed and taught introductory through
    advanced-level programming courses in IDL and ENVI, with over 150
    classes and 1500 students taught to excellent reviews. Primary
    customers included NASA, NOAA, Los Alamos National Lab, Lawrence
    Livermore National Lab, Sandia National Lab, JPL, University of
    Colorado, Raytheon, and USGS.
  \item Created domain-specific courses on signal processing, medical
    image processing, and scientific programming with IDL.
  \item Worked on consulting projects programming in IDL, ENVI, C,
    Java, and Fortran. Responsible for project estimation, project
    design and writing of technical specifications.
  \item Conducted shoulder-to-shoulder consulting and problem-solving
    with customers.
  \item Acted as an internal technical resource for Tech Support,
    Sales, Engineering, and Services.
\end{compactitem}

\affiliation[Postdoctoral Research Associate] {University of Colorado
  at Boulder, Boulder, CO} {2003}
\begin{compactitem}[\itembullet]
  \item Extended doctoral research during a three-month sabbatical
    from Research Systems, Inc.
  \item Coauthored one journal article.
\end{compactitem}

\affiliation[Graduate Research Assistant]{University of Colorado at
  Boulder, Boulder, CO}{1995--1999}
\begin{compactitem}[\itembullet]
  \item Performed research into the nature of turbulent kinetic energy
    dissipation in a frontal zone.
  \item Developed, coded (in IDL, Fortran and C) and employed
    algorithms for computing statistics from sonic and hot-wire
    anemometer data taken in the atmospheric boundary layer.
  \item Presented the first high-wavenumber measurements of
    dissipation rate in a frontal zone, as well as comparisons with
    techniques for estimating dissipation rate at lower wavenumbers,
    at three academic conferences.
  \item Coauthored two journal articles with academic advisor.
\end{compactitem}

\affiliation[Graduate Research Assistant]{University of Minnesota
  Twin Cities, Minneapolis, MN}{1995}
\begin{compactitem}[\itembullet]
  \item Started research on modeling space-time rainfall distributions
    with the NCAR MM5 community model at the Saint Anthony Falls
    Laboratory.
  \item Took two quarters of classes on environmental modeling and
    statistics in the Department of Civil Engineering.
\end{compactitem}

\affiliation[Associate Research Scientist]{University of Wisconsin
  Space Science and Engineering Center, Madison, WI}{1994}
\begin{compactitem}[\itembullet]
  \item Built and tested electronics components for automatic weather
    stations as a staff member of the Antarctic Meteorological
    Research Center.
  \item Programmed and tested Campbell Scientific CR10 dataloggers.
\end{compactitem}

\affiliation[Graduate Research Assistant]{Penn State, University Park,
  PA}{1993--1994}
\begin{compactitem}[\itembullet]
  \item Performed a fluid dynamics laboratory experiment at the EPA
    Fluid Modeling Facility in Research Triangle Park, NC.
\end{compactitem}

\affiliation[Graduate Teaching Assistant]{Penn State, University Park,
  PA}{1992--1993}
\begin{compactitem}[\itembullet]
  \item Taught two semesters (six sections) of a one-credit
    undergraduate meteorology lab.
\end{compactitem}

\affiliation[Undergraduate Research Assistant]{University of Wisconsin
  Space Science and Engineering Center, Madison, WI} {1992}
\begin{compactitem}[\itembullet]
  \item NSF Research Experience for Undergraduates (REU) student with
    Professor Charles Stearns, Department of Atmospheric and Oceanic
    Sciences.
  \item Worked four weeks at the Greenland Ice Sheet Project 2 (GISP2)
    base camp to repair and maintain three automatic weather stations.
  \item Presented work at the Sixth National Conference on
    Undergraduate Research at the University of Minnesota Twin Cities.
\end{compactitem}
