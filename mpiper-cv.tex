%-----------------------------------------------------------------------------
% CV
%-----------------------------------------------------------------------------

\documentclass[letterpaper]{resume}

\usepackage[numbers,sort&compress]{natbib}  % for bibliography
\renewcommand{\bibsection}[0]{}  % removing bibliography header
\usepackage{url}

\begin{document}

%\begin{center}
%\large\textsc{Cirriculum Vitae}
%\end{center}
%\vspace{1.0em}

\author{Mark Daniel Piper}
\email{mark.piper@colorado.edu}
\phone{1-303-492-6068}
\webpage{http://instaar.colorado.edu/people/mark-piper}
\orcid{http://orcid.org/0000-0001-6418-277X}
\github{@mdpiper}
\streetaddress{Institute of Arctic and Alpine Research \\
Community Surface Dynamics Modeling System \\
University of Colorado Boulder \\
4001 Discovery Drive}
\citystatezip{Boulder, CO 80303}

\maketitle

%-----------------------------------------------------------------------------

\section{Education}
\vspace{0.5em}

\affiliation[Ph.D., Astrophysical, Planetary, and Atmospheric Sciences]
            {University of Colorado (Boulder, CO, USA)}
            {1995--2001}
            \par Advisor: William Blumen
            \par Thesis: \textit{The effects of a frontal passage on
              fine-scale nocturnal boundary layer turbulence}

\affiliation[M.S., Meteorology]
            {Penn State (University Park, PA, USA)}
            {1992--1994}
            \par Advisor: John Wyngaard
            \par Thesis: \textit{Top-down, bottom-up diffusion
              experiments in a water convection tank}

\affiliation[B.S., Mathematics]
            {University of Wisconsin (Madison, WI, USA)}
            {1988--1992}

%-----------------------------------------------------------------------------

\section{Employment}
\vspace{0.5em}

\affiliation[Research Associate]{University of Colorado, Boulder,
  CO}{2013--present}
\begin{compactitem}[\itembullet]
  \item Research Software Engineer in the Community Surface Dynamics
    Modeling System (CSDMS) Integration Facility at the Institute of
    Arctic and Alpine Research (INSTAAR).
  \item Developed Java-based front-end for the CSMDS Web Modeling Tool (WMT).
  \item Developed tools for building and installing CSDMS software on
    RPM-based Linux distributions.
  \item Wrote Python wrappers for Dakota and CSDMS components.
  \item Developed Java implementation of the Basic Model Interface (BMI).
  \item Co-I on two funded proposals: NSF-1503559: Toward a Tiered Permafrost Modeling 
    Cyberinfrastructure; NASA-14-CMAC14-0014: A Permafrost Benchmark System 
    to Evaluate Permafrost Models
  \item Internal programming resource for the broader CSDMS community.
\end{compactitem}

\affiliation[Product Manager]{Exelis Visual Information Solutions
  (formerly Research Systems, Inc. (RSI)), Boulder, CO}{2012--2013}
\begin{compactitem}[\itembullet]
  \item Technical product manager, product owner and technology
    evangelist for the scientific data analysis and visualization
    language IDL. In concert with the lead developer, tasked with
    guiding the overall direction of the language, including
    determining what new capabilities to add and prioritizing what
    bugs to fix.
  \item Responsibilities include performing market research and
    competitive analyses (versus, primarily, Python and MATLAB),
    gathering user feedback though onsite visits (particularly at
    NASA, NOAA and U.S. National Labs) and attendance at scientific
    conferences, participating in the design of the IDL and ENVI
    (geospatial image analysis application) APIs, writing user
    stories, developing requirements, managing the product backlog and
    effectively communicating information to the Engineering team.
  \item Act as an internal technical resource for Sales, Marketing,
    Tech Support, Services and Engineering.
\end{compactitem}

\affiliation[Solutions Engineer]{Exelis Visual Information
  Solutions, Boulder, CO}{2010--2012}
\begin{compactitem}[\itembullet]
  \item Acted as product owner for IDL. Duties included writing user
    stories and translating user feedback to product development,
    including working with the IDL product manager and lead developer
    to prioritize development for future releases. Responsible for
    internal and external technical IDL product launch communication,
    using presentations, whitepapers and code to demonstrate new
    features and functionality.
  \item Acted as technology evangelist for IDL. Attended scientific
    conferences to meet customers, obtain product feedback and conduct
    market research. Engaged in user outreach by developing and
    presenting quarterly live seminars and webinars on topics ranging
    from numerical analysis techniques to file formats to
    visualization techniques in IDL. Created and maintained a blog at
    {\url{idldatapoint.wordpress.com}}, highlighting solutions
    to current problems with IDL. Engaged with users through Twitter
    (@TheIDLGuy). Wrote programmatic examples for customers. Acted as
    a resource to the IDL user community through communication with
    long-time customers and through involvement with the
    \url{comp.lang.idl-pvwave} Usenet newsgroup.
  \item Continued to act as an internal technical resource for Sales,
    Marketing, Tech Support, Services, Engineering and Product
    Management.
\end{compactitem}

\affiliation[Professional Services Engineer]{Exelis Visual Information
  Solutions, Boulder, CO}{1999--2010}
\begin{compactitem}[\itembullet]
  \item Designed, developed and taught introductory through
    advanced-level programming courses in IDL and ENVI, with over 150
    classes and 1500 students taught to excellent reviews. Primary
    customers included NASA, NOAA, Los Alamos National Lab, Lawrence
    Livermore National Lab, Sandia National Lab, JPL, University of
    Colorado, Raytheon and USGS. Created domain-specific courses on
    signal processing, medical image processing and scientific
    programming with IDL.
  \item Worked on consulting projects programming in IDL, C, Java and
    Fortran. Responsible for project estimation, project design and
    writing of technical specifications. Conducted
    shoulder-to-shoulder consulting and problem-solving with
    customers.
  \item Acted as an internal technical resource to Tech Support,
    Sales, Engineering and Services.
\end{compactitem}

\affiliation[Postdoctoral Research Associate] {University of Colorado
  at Boulder, Boulder, CO} {2003}
\begin{compactitem}[\itembullet]
\item Extended doctoral research during a three-month sabbatical from
  Research Systems, Inc. Coauthored one journal article.
\end{compactitem}

\affiliation[Graduate Research Assistant]{University of Colorado at
  Boulder, Boulder, CO}{1995--1999}
\begin{compactitem}[\itembullet]
  \item Performed research into the nature of turbulent kinetic energy
    dissipation in a frontal zone.
  \item Developed, coded (in IDL, Fortran and C) and employed
    algorithms for computing statistics from sonic and hot-wire
    anemometer data taken in the atmospheric boundary layer.
  \item Presented the first high-wavenumber measurements of
    dissipation rate in a frontal zone, as well as comparisons with
    techniques for estimating dissipation rate at lower wavenumbers,
    at three academic conferences. Coauthored two journal articles.
\end{compactitem}

\affiliation[Graduate Research Assistant]{University of Minnesota
  Twin Cities, Minneapolis, MN}{1995}
\begin{compactitem}[\itembullet]
  \item Started research on modeling space-time rainfall distributions
    with the MM5 community model at the Saint Anthony Falls
    Laboratory. Took coursework on environmental modeling and
    statistics in the Department of Civil Engineering.
\end{compactitem}

\affiliation[Associate Research Scientist]{University of Wisconsin
  Space Science and Engineering Center, Madison, WI}{1994}
\begin{compactitem}[\itembullet]
  \item Built, assembled and tested electronics components for
    automatic weather stations as a staff member of the Antarctic
    Meteorological Research Center. Programmed and tested Campbell
    Scientific CR10 dataloggers.
\end{compactitem}

\affiliation[Graduate Research Assistant]{Penn State, University Park,
  PA}{1993--1994}
\begin{compactitem}[\itembullet]
  \item Developed and built a fluid dynamics laboratory experiment at
    the EPA Fluid Modeling Facility in Research Triangle Park, NC.
\end{compactitem}

\affiliation[Graduate Teaching Assistant]{Penn State, University Park,
  PA}{1992--1993}
\begin{compactitem}[\itembullet]
  \item Taught six sections of an introductory undergraduate
    meteorology lab.
\end{compactitem}

\affiliation[Undergraduate Research Assistant]{University of Wisconsin
  Space Science and Engineering Center, Madison, WI} {1992}
\begin{compactitem}[\itembullet]
  \item NSF Research Experience for Undergraduates (REU) student with
    Professor Charles Stearns, Department of Atmospheric and Oceanic
    Sciences.
  \item Worked four weeks at the Greenland Ice Sheet Project 2 (GISP2)
    base camp to repair and maintain three automatic weather stations.
\end{compactitem}

%-----------------------------------------------------------------------------

\section{Bibliography}
\vspace{0.5em}

\nocite{*}
\bibliographystyle{agufull08}
\bibliography{mpiper-citations}

%-----------------------------------------------------------------------------

\section{Invited talks, workshops, and short courses}
\vspace{0.5em}

\begin{enumerate}[{[}1{]}]

  \item \textit{CSDMS Software Carpentry Bootcamp}, (short course with
    M. Perignon), CSDMS Annual Meeting, Boulder, CO, May 2018

  \item \textit{BMI Live}, (short course with E. Hutton),
    CSDMS Annual Meeting, Boulder, CO, May 2018

  \item \textit{BMI Hackathon}, (hackathon convened with E. Hutton),
    CSDMS Annual Meeting, Boulder, CO, May 2018

  \item \textit{Community Cyberinfrastructure for Modeling Earth-Surface
    Processes}, (invited talk), Coupling Surface and Tectonic
    Processes (CTSP) Workshop, Boulder, CO, April 2018

  \item \textit{CSDMS Software Carpentry Bootcamp}, (short course with
    M. Perignon), CSDMS Annual Meeting, Boulder, CO, May 2017

  \item \textit{BMI Live}, (short course with E. Hutton),
    CSDMS Annual Meeting, Boulder, CO, May 2017

  \item \textit{BMI Hackathon}, (hackathon convened with E. Hutton),
    CSDMS Annual Meeting, Boulder, CO, May 2018

  \item \textit{CSDMS Software Carpentry Bootcamp}, (short course with
    M. Perignon), CSDMS Annual Meeting, Boulder, CO, May 2016

  \item \textit{BMI Live}, (short course with E. Hutton),
    CSDMS Annual Meeting, Boulder, CO, May 2016

  \item \textit{WMT and the Dakota Iterative Systems Analysis Toolkit},
    (short course), CSDMS Annual Meeting, Boulder, CO, May 2015

  \item \textit{ATOC IDL Seminar}, (short course), Department of
    Atmospheric and Oceanic Sciences, University of Colorado, Boulder,
    CO, January 2015.

  \item \textit{WMT: The CSDMS Web Modeling Tool}, (short course), CSDMS
    Annual Meeting, Boulder, CO, May 2014

  \item \textit{A Gallery of Scientific Visualizations}, {VISualize} 2013,
    Washington, DC, June 2013

  \item \textit{IDL Update}, ENVI and IDL User Group Meeting at the 2012
    AGU Fall Meeting, San Francisco, CA, December 2012

  \item \textit{Using IDL with Suomi NPP VIIRS Data}, {VISualize} 2012,
    Washington, DC, June 2012

  \item \textit{IDL 8 Overview}, 2010 AGU IDL User Group Meeting, San
    Francisco, CA, December 2010

  \item \textit{Introduction to Scientific Programming with IDL}, 2010 IDL
    User Group Meeting, Boulder, CO, February 2010

  \item \textit{Using the IDL Workbench}, 2008 IDL User Group Meeting,
    Boulder, CO, October 2008

\end{enumerate}

%-----------------------------------------------------------------------------

\section{Conference presentations}
\vspace{0.5em}

\begin{enumerate}[{[}1{]}]

  \item Piper, M., E. Hutton, and J. Syvitski (2016), A Python
    Interface for the Dakota Iterative Systems Analysis Toolkit, AGU
    Fall Meeting, San Francisco, CA, (talk).

  \item Piper, M., E. Hutton, I. Overeem, and J. Syvitski (2015),
    {WMT}: The {CSDMS} Web Modeling Tool, AGU Fall Meeting, San
    Francisco, CA, (poster).

  \item Piper, M. (2013), A Rapid Cloud Mask Algorithm for {Suomi}
    {NPP} {VIIRS} {Imagery} {EDRs}, NOAA Satellite Conference, College
    Park, MD, (poster).

  \item Piper, M. (2012), {A Rapid Cloud Mask Algorithm for {Suomi}
    {NPP} {VIIRS} {Imagery} {EDRs}}, AGU Fall Meeting, San Francisco,
    CA, (poster).

  \item Piper, M. (2012), {Using IDL with {Suomi} {NPP} {VIIRS} Data},
    HDF and HDF-EOS Workshop XV, Riverdale, MD, (talk).

  \item Piper, M., B. Justice, A. Borsholm, and A.T. Harris (2011),
    Use of Open Geospatial Consortium (OGC) Standards to Disseminate
    and Access Scientific Data, AGU Fall Meeting, San Francisco, CA,
    (poster).

  \item Piper, M. and J.K. Lundquist (2004), Surface-Layer Turbulence
    during a Frontal Passage, AMS 16th Symposium on Boundary Layers
    and Turbulence, Portland, ME, (talk).

  \item Lundquist, J.K., M. Piper, and B. Kosovi{\'c} (2004), TKE
    Budgets and Dissipation Length in Disturbed Boundary Layers, AMS
    16th Symposium on Boundary Layers and Turbulence, Portland, ME,
    (talk).

  \item Piper, M, W. Blumen, and R.L. Grossman (1999), Analysis
    of Dissipation in Fronts, AMS 13th Symposium on Boundary Layers
    and Turbulence, Dallas, TX, (poster).

  \item Piper, M., W. Blumen, and N. Gamage (1998), Hotwire
    Anemometer Measurements of Dissipation Rate in Surface-Layer
    Turbulence, AMS 10th Symposium on Meteorological Observations
    and Instrumentation, Phoenix, AZ, (poster).

  \item Piper, M., W. Blumen, and N. Gamage (1997), Sampling of
    Coherent Structures from Bursts of Dissipation Rate, AMS 12th
    Symposium on Boundary Layers and Turbulence, Vancouver, BC,
    (poster).

  \item Piper, M., W. Blumen, and N. Gamage (1997), Effects of a Dry
    Cold Front Passage on Surface-Layer Turbulence, AMS 12th Symposium
    on Boundary Layers and Turbulence, Vancouver, BC, (talk).

  \item Piper, M., J.C. Wyngaard, W.H. Snyder, and R.E. Lawson (1995),
    Convection Tank Experiments on Top-Down, Bottom-Up Diffusion, AMS
    11th Symposium on Boundary Layers and Turbulence, Charlotte, NC,
    (talk).

\end{enumerate}

%-----------------------------------------------------------------------------

\section{Software products}
\vspace{0.5em}

\begin{enumerate}[{[}1{]}]

  \item Piper, M., The Permafrost Benchmark System (v1.0),
    \url{https://permamodel.github.io/pbs}.

  \item Piper, M., BMI for the ILAMB benchmarking toolkit (v0.1.1),
    \url{https://github.com/permamodel/bmi-ilamb}.

  \item Piper, M., Dakotathon (v0.4.1),
    \url{https://github.com/csdms/dakota}.

  \item Piper, M., Web Modeling Tool Client (v1.1),
    \url{https://github.com/csdms/wmt-client}.

  \item Hutton, E., and M. Piper, Web Modeling Tool Executor (v0.1),
    \url{https://github.com/csdms/wmt-exe}.

  \item Hutton, E., and M. Piper, Web Modeling Tool Server and API (v1.0),
    \url{https://github.com/csdms/wmt}.

  \item Hutton, E., and M. Piper, Web Modeling Tool Metadata,
    \url{https://github.com/csdms/wmt-metadata}.

  \item Piper, M., Web Modeling Tool Selector (v0.1),
    \url{https://github.com/csdms/wmt-selector}.

  \item Hutton, E., and M. Piper, BMI Python Bindings (v0.2),
    \url{https://github.com/bmi-forum/bmi-python}.

  \item Piper, M., BMI Java Bindings (v0.1),
    \url{https://github.com/csdms/bmi-java}.

  \item Piper, M., BMI Fortran Bindings (v0.3),
    \url{https://github.com/csdms/bmi-fortran}.

  \item Piper, M., CSDMS RPM Repository,
    \url{https://csdms.colorado.edu/repo}.

  \item Piper, M., VIIRS I-Band Cloud Mask (v1.0),
    \url{https://github.com/mdpiper/viirs-cloudmask}.

  \item Piper, M., IDL GRIB Helper Library (v1.0),
    \url{https://github.com/mdpiper/grib-helper-lib}.

\end{enumerate}

%-----------------------------------------------------------------------------

\section{Technical skills}
\vspace{0.5em}

\begin{itemize}

  \item \textbf{Programming languages:}
    Python, Java, Fortran, IDL, bash, C, MATLAB, Javascript

  \item \textbf{Software development tools:}
    Git, Subversion, CMake, make, conda, Eclipse, Emacs, Vi/m, Slurm,
    Torque, gcc, gfortran, bash, tcsh, Ant, Nose, Sphinx, Jupyter
    Notebook, Jupyterhub, \LaTeX, Markdown, reStructuredText,
    MediaWiki, {Travis CI}, Coveralls, rpmbuild, Confluence, JIRA,
    Camtasia, Word, Powerpoint, Excel

  \item \textbf{Operating systems:}
    Linux, macOS, Windows

\end{itemize}

%-----------------------------------------------------------------------------

\section{Professional societies}
\vspace{0.5em}

\begin{itemize}
  \item American Geophysical Union (2010--present) 
  \item Sigma Xi (1992--present)
  \item American Meteorological Society (1990--present)
\end{itemize}

%-----------------------------------------------------------------------------

\section{Professional activities and service}
\vspace{0.5em}

\begin{itemize}

  \item Panelist, NSF CISE/OAC (2018)

  \item Reviewer:
    \begin{compactitem}
    \item \textit{Journal of Open Source Software} (2018)
    \item \textit{Basin Research} (2017)
    \item \textit{Computers \& Geosciences} (2016)
    \end{compactitem}

  \item Judge, CSDMS Student Modeler Award (2014---present)

  \item Judge, CSDMS Student Scholarship Awards (2014---present)

  \item Program committee member, HDF and HDF-EOS Workshop XVI (2013)

  \item Judge, Colorado Undergraduate Space Research Symposium (2008)

\end{itemize}

%-----------------------------------------------------------------------------

\section{Honors and awards}
\vspace{0.5em}

\begin{itemize}

  \item Graduate School Fellowship, University of Colorado at Boulder
    (1995--1998)

  \item Second place, student oral presentation, AMS 12th Symposium on
    Boundary Layers and Turbulence (1997)

  \item Hans Neuberger Award for Outstanding Teaching, Department of
    Meteorology, Penn State (1993)

\end{itemize}

%-----------------------------------------------------------------------------

\section{Certifications}
\vspace{0.5em}

\begin{itemize}
  \item Product Management Certification, Pragmatic Marketing (2012)
\end{itemize}

%-----------------------------------------------------------------------------

\end{document}
